
\documentclass{article}
\usepackage[utf8]{inputenc}
\usepackage[spanish]{babel}
\usepackage{subfigure} % subgráficos
%\usepackage{graphicx} % gráficos

\usepackage{graphicx}
\usepackage{listings}
\usepackage[T1]{fontenc}
%\usepackage{subcaption}


\usepackage{float}
\usepackage{color}
\usepackage{natbib}
\usepackage{xcolor}
\usepackage{csquotes}
\usepackage{booktabs}

\usepackage {hyperref}
 \hypersetup{breaklinks=true,colorlinks=true,
        linkcolor=black,citecolor=black,urlcolor=black}

\definecolor{dkgreen}{rgb}{0,0.6,0}
\definecolor{gray}{rgb}{0.5,0.5,0.5}
\definecolor{mauve}{rgb}{0.58,0,0.82}
\definecolor{gray97}{gray}{.97}
\definecolor{gray75}{gray}{.75}
\definecolor{gray45}{gray}{.45}

\lstset{
%frame=tb,
 language=Python,
 backgroundcolor=\color{gray97},
 captionpos=b,
 keepspaces=true, 
 numbers=left,
 numbersep=5pt, 
  firstnumber=1, 
  stepnumber=1, 
  showstringspaces=false,
  tabsize=1,
% numberstyle=\small\color{migris}
 stringstyle=\color{mimalva}, 
 commentstyle=\color{miverde},
 stepnumber=1,  
 frame=single,
  escapeinside={\%*}{*)},  
  language=Python,
  aboveskip=3mm,
  belowskip=3mm,
  showstringspaces=false,
  columns=flexible,
  basicstyle={\small\ttfamily},
  %numbers=none,%esto estaba evitando que saliera la columna con los numeros en la parte del código en el documento
  %numberstyle=\tiny\color{gray},
  numberstyle=\footnotesize,
  keywordstyle=\color{blue},
 % commentstyle=\color{dkgreen},
  stringstyle=\color{mauve},
  breaklines=true,
  breakatwhitespace=false,
  tabsize=3
}

\begin{document}
\title{Tarea 5 Optimización de Flujo en Redes}
\author{1985274}
\date{\today}
\maketitle 

\maketitle

En este trabajo se presenta un análisis en el que se ha seleccionado un algoritmo generador de grafos y a su vez se ha escogido una implementación de algoritmo de flujo máximo. Para esto se ha utilizado el lenguaje \textbf{Python} en su versión 3.7 \cite{python}, el editor de código Spyder en su versión 3.3.1 y el editor Texmaker 5.0.2 para redactar el documento. Se utilizan además las librerías \textbf{networkX} 1.5 \cite{networkx}, \textbf{Matplotlib} \cite{matplotlib}, \textbf{Numpy}  \cite{numpy}, la librería \textbf{Panda} \cite{panda}, la librería \textbf{Scipy} \cite{scipy}, la librería \textbf{Seaborn} \cite{seaborn} para graficar, la librería \textbf{Researchcpy} \cite{researchcpy} para producir pandas \texttt{Dataframe} de pruebas estadísticas y la librería \textbf{Pingouin} \cite{pingouin} para realizar el análisis de varianza de un factor (ANOVA). 

\section{Algoritmo generador de grafo}  
Se utiliza el generador \textit{Watts Strogatz graph} y como algoritmo de flujo máximo: \textit{maximun flow value}.

\begin{itemize}
 
  \item Algoritmo \textit{Watts Strogatz graph}: Este es un generador aleatorio que recibe cuatro parámetros que son: $n$ el número de nodos, $k$ cada nodo se une con sus $k$ vecinos en una topología de anillo, $p$ que es la probabilidad de volver a establecer una arista, $seed$ que es la semilla o indicador del estado de generación de números aleatorios. Este algoritmo primero crea un anillo sobre $n$ nodos y luego cada nodo en el anillo es conectado con sus vecinos más cercanos. Luego se crean accesos directos reemplazando algunas aristas. Se seleccionó este algoritmo porque es posible representar una red de transporte de una terminal aérea de un país con varias aerolíneas, donde cada una enlaza este país con otros países de ida y vuelta, por lo que se obtiene un grafo no dirigido. En este caso los países son los nodos y las aristas son las aerolíneas.  

\end{itemize}

\section{Algoritmo de flujo máximo}  

\begin{itemize}
  \item Algoritmo \textit{maximum flow value}: Esta función calcula el valor del flujo máximo a costo mínimo entre la fuente y el sumidero en grafos capacitados. El algoritmo recibe varios parámetros como son: $G$ que es el grafo, $s$ que es el nodo origen o fuente para el flujo, $t$ que es el nodo destino o sumidero para el flujo, $capacidad$ que es el parámetro que indica la cantidad de flujo que puede soportar la arista. Al aplicar este algoritmo se obtiene un diccionario que contiene el valor del flujo que pasó por cada arista. 
  Se generaron 5 grafos de 20 nodos cada uno, y sobre ellos se ejecutó al algoritmo de flujo máximo \textit{maximum flow value}.     
   
\end{itemize}

\subsection{Fragmento del conjunto de datos obtenido}

% Table generated by Excel2LaTeX from sheet 'timesCopiaKOk'
\begin{table}[htbp]
  \centering
  \caption{Fragmento del conjunto de datos obtenido}
  \resizebox{\textwidth}{!}{
    \begin{tabular}{rrrrrrrrrrr}
    \toprule
    \hline
    \textbf{Grafo} & \multicolumn{1}{l} {Fuente} & \multicolumn{1}{l}{Sumidero} & \multicolumn{1}{l}{Media} & \multicolumn{1}{l}{Mediana} & \multicolumn{1}{l}{Std} & \multicolumn{1}{l}{FlujMax} & \multicolumn{1}{l}{Grado} & \multicolumn{1}{l}{CoefAg} & \multicolumn{1}{l}{CentCe} & \multicolumn{1}{l}{CenCar}\\
    \hline 
    1     & 4     & 0     & 0.022 & 0.0041 & 0.0231 & 41    & 4     & 0.5   & 0.365 & 0.045       \\
    1     & 4     & 1     & 0.023 & 0.0065 & 0.0229 & 41    & 4     & 0.5   & 0.365 & 0.045       \\
    1     & 4     & 2     & 0.022 & 0.0065 & 0.0219 & 41    & 4     & 0.5   & 0.365 & 0.045       \\
    1     & 4     & 3     & 0.024 & 0.0068 & 0.0231 & 41    & 4     & 0.5   & 0.365 & 0.045       \\
    1     & 4     & 5     & 0.021 & 0.0041 & 0.0215 & 37    & 4     & 0.5   & 0.365 & 0.045       \\
    1     & 4     & 6     & 0.021 & 0.0066 & 0.0189 & 41    & 4     & 0.5   & 0.365 & 0.045       \\
    1     & 4     & 7     & 0.021 & 0.0051 & 0.0204 & 41    & 4     & 0.5   & 0.365 & 0.045       \\
    1     & 4     & 8     & 0.020 & 0.0054 & 0.0205 & 41    & 4     & 0.5   & 0.365 & 0.045       \\
    1     & 4     & 9     & 0.021 & 0.0085 & 0.0191 & 41    & 4     & 0.5   & 0.365 & 0.045       \\
    1     & 4     & 10    & 0.021 & 0.0059 & 0.0200 & 41    & 4     & 0.5   & 0.365 & 0.045       \\
    1     & 4     & 11    & 0.024 & 0.0044 & 0.0252 & 41    & 4     & 0.5   & 0.365 & 0.045       \\
    1     & 4     & 12    & 0.024 & 0.0060 & 0.0234 & 41    & 4     & 0.5   & 0.365 & 0.045       \\
    1     & 4     & 13    & 0.022 & 0.0072 & 0.0214 & 39    & 4     & 0.5   & 0.365 & 0.045       \\
    1     & 4     & 14    & 0.021 & 0.0074 & 0.0200 & 41    & 4     & 0.5   & 0.365 & 0.045       \\
    1     & 4     & 15    & 0.024 & 0.0091 & 0.0225 & 41    & 4     & 0.5   & 0.365 & 0.045       \\
    1     & 4     & 16    & 0.021 & 0.0058 & 0.0204 & 41    & 4     & 0.5   & 0.365 & 0.045       \\
    1     & 4     & 17    & 0.021 & 0.0062 & 0.0201 & 41    & 4     & 0.5   & 0.365 & 0.045       \\
    1     & 4     & 18    & 0.021 & 0.0070 & 0.0207 & 38    & 4     & 0.5   & 0.365 & 0.045       \\
    1     & 4     & 19    & 0.024 & 0.0073 & 0.0230 & 41    & 4     & 0.5   & 0.365 & 0.045       \\  
    \hline
    \bottomrule
    \end{tabular}%
    }
  \label{tab:addlabel}%
\end{table}%


\subsection{Código}

%\lstinputlisting[language=Python, firstline=11, lastline=110]{junto.py}

\subsection{Diagrama de los grafos generados}
\begin{itemize}
 
  \item A continuación se muestran los diagramas de los 5 grafos que se obtuvieron a partir del generador mencionado anteriormente. Todos los grafos se generaron con 20 nodos. En las imágenes del grafo se denota por colores cuál es el mejor par nodo fuente sumidero. El nodo de color amarillo denota que es un nodo fuente y el nodo azul denota que es un nodo destino o sumidero. De igual modo, el flujo se denota por colores en las aristas y la capacidad se denota por el ancho de la arista, en donde a mayor ancho de la arista mayor es la capacidad de la arista. En el caso del flujo, el color verde denota que existe flujo por esa arista, y se presenta desde un tono más claro a un tono más oscuro, para denotar que el flujo va en aumento,el color negro denota que por esa arista No transita ningún flujo.  
  
  Para cada grafo se evaluaron 6 características, como son: Grado de los nodos, el coeficiente de agrupamiento, el coeficiente de centralidad de cercanía, el coeficiente de centralidad de carga, el coeficiente de excentricidad y el coeficiente pagerank. 
\end{itemize}

Diagramas de los 5 grafos con el mejor par fuente - sumidero 

\begin{figure}[H]
\subfigure[\textit{Fig 1}]{\includegraphics[scale=0.25]{fig7delGrafo7.eps}}
\subfigure[\textit{Fig 2}]{\includegraphics[scale=0.25]{fig8delGrafo8.eps}}
\subfigure[\textit{Fig 3}]{\includegraphics[scale=0.25]{fig9delGrafo9.eps}}
\subfigure[\textit{Fig 4}]{\includegraphics[scale=0.25]{fig10delGrafo10.eps}}
\subfigure[\textit{Fig 5}]{\includegraphics[scale=0.25]{fig11delGrafo11.eps}}
\caption{Ejemplos de grafos}
\label{fig2} 
\end{figure}


Diagramas de los 5 grafos con el peor par fuente - sumidero 

\begin{figure}[H]
\subfigure[\textit{Fig 1}]{\includegraphics[scale=0.25]{fig7delPeorParFuenteSumideroGrafo7.eps}}
\subfigure[\textit{Fig 2}]{\includegraphics[scale=0.25]{fig8delPeorParFuenteSumideroGrafo8.eps}}
\subfigure[\textit{Fig 3}]{\includegraphics[scale=0.25]{fig9delPeorParFuenteSumideroGrafo9.eps}}
\subfigure[\textit{Fig 4}]{\includegraphics[scale=0.25]{fig10delPeorParFuenteSumideroGrafo10.eps}}
\subfigure[\textit{Fig 5}]{\includegraphics[scale=0.25]{fig11delPeorParFuenteSumideroGrafo11.eps}}
\caption{Ejemplos de grafos}
\label{fig2} 
\end{figure}

\section{Histogramas}
\begin{itemize} 
  \item A continuación se exponen los histogramas de cada uno de los 5 grafos.
  
  
\begin{figure}[H]
\subfigure[\textit{Fig 1}]{\includegraphics[scale=0.25]{histCentralidadCartimesdelGrafo7csv.eps}}
\subfigure[\textit{Fig 2}]{\includegraphics[scale=0.25]{histCentralidadCartimesdelGrafo8csv.eps}}
\subfigure[\textit{Fig 3}]{\includegraphics[scale=0.25]{histCentralidadCartimesdelGrafo9csv.eps}}
\subfigure[\textit{Fig 4}]{\includegraphics[scale=0.25]{histCentralidadCartimesdelGrafo10csv.eps}}
\subfigure[\textit{Fig 5}]{\includegraphics[scale=0.25]{histCentralidadCartimesdelGrafo11csv.eps}}
\caption{Histogramas que representan la característica de Centralidad de carga}
\label{fig2} 
\end{figure}  

\begin{figure}[H]
\subfigure[\textit{Fig 1}]{\includegraphics[scale=0.25]{histCentralidadCetimesdelGrafo7csv.eps}}
\subfigure[\textit{Fig 2}]{\includegraphics[scale=0.25]{histCentralidadCetimesdelGrafo8csv.eps}}
\subfigure[\textit{Fig 3}]{\includegraphics[scale=0.25]{histCentralidadCetimesdelGrafo9csv.eps}}
\subfigure[\textit{Fig 4}]{\includegraphics[scale=0.25]{histCentralidadCetimesdelGrafo10csv.eps}}
\subfigure[\textit{Fig 5}]{\includegraphics[scale=0.25]{histCentralidadCetimesdelGrafo11csv.eps}}
\caption{Histogramas que representan la característica de Centralidad de cercanía}
\label{fig2} 
\end{figure}  

\begin{figure}[H]
\subfigure[\textit{Fig 1}]{\includegraphics[scale=0.25]{histCoefAgruptimesdelGrafo7csv.eps}}
\subfigure[\textit{Fig 2}]{\includegraphics[scale=0.25]{histCoefAgruptimesdelGrafo8csv.eps}}
\subfigure[\textit{Fig 3}]{\includegraphics[scale=0.25]{histCoefAgruptimesdelGrafo9csv.eps}}
\subfigure[\textit{Fig 4}]{\includegraphics[scale=0.25]{histCoefAgruptimesdelGrafo10csv.eps}}
\subfigure[\textit{Fig 5}]{\includegraphics[scale=0.25]{histCoefAgruptimesdelGrafo11csv.eps}}
\caption{Histogramas que representan la característica de Coeficiente de Agrupamiento}
\label{fig2} 
\end{figure}  

\begin{figure}[H]
\subfigure[\textit{Fig 1}]{\includegraphics[scale=0.25]{histExcentricidadtimesdelGrafo7csv.eps}}
\subfigure[\textit{Fig 2}]{\includegraphics[scale=0.25]{histExcentricidadtimesdelGrafo8csv.eps}}
\subfigure[\textit{Fig 3}]{\includegraphics[scale=0.25]{histExcentricidadtimesdelGrafo9csv.eps}}
\subfigure[\textit{Fig 4}]{\includegraphics[scale=0.25]{histExcentricidadtimesdelGrafo10csv.eps}}
\subfigure[\textit{Fig 5}]{\includegraphics[scale=0.25]{histExcentricidadtimesdelGrafo11csv.eps}}
\caption{Histogramas que representan la característica de Excentricidad}
\label{fig2} 
\end{figure} 

\begin{figure}[H]
\subfigure[\textit{Fig 1}]{\includegraphics[scale=0.25]{histGradotimesdelGrafo7csv.eps}}
\subfigure[\textit{Fig 2}]{\includegraphics[scale=0.25]{histGradotimesdelGrafo8csv.eps}}
\subfigure[\textit{Fig 3}]{\includegraphics[scale=0.25]{histGradotimesdelGrafo9csv.eps}}
\subfigure[\textit{Fig 4}]{\includegraphics[scale=0.25]{histGradotimesdelGrafo10csv.eps}}
\subfigure[\textit{Fig 5}]{\includegraphics[scale=0.25]{histGradotimesdelGrafo11csv.eps}}
\caption{Histogramas que representan la característica de Grado}
\label{fig2} 
\end{figure}

\begin{figure}[H]
\subfigure[\textit{Fig 1}]{\includegraphics[scale=0.25]{histPageRanktimesdelGrafo7csv.eps}}
\subfigure[\textit{Fig 2}]{\includegraphics[scale=0.25]{histPageRanktimesdelGrafo8csv.eps}}
\subfigure[\textit{Fig 3}]{\includegraphics[scale=0.25]{histPageRanktimesdelGrafo9csv.eps}}
\subfigure[\textit{Fig 4}]{\includegraphics[scale=0.25]{histPageRanktimesdelGrafo10csv.eps}}
\subfigure[\textit{Fig 5}]{\includegraphics[scale=0.25]{histPageRanktimesdelGrafo11csv.eps}}
\caption{Histogramas que representan la característica de \textit{PageRank}}
\label{fig2} 
\end{figure}

\end{itemize}

\begin{itemize}
 
  \item Como se puede apreciar en las gráfias anteriores los características no siguen una distribución normal, por lo que se utiliza la mediana para hacer el analisis estadístico. Se procedió a realizar una normalización y agrupamiento para categorizar en Bajo, medio y Alto, obteniéndose 3 grupos.
\end{itemize}

\section{Análisis de varianza ANOVA}  
Luego se realizó un análisis de varianza (ANOVA) de un factor sobre estos datos, con el objetivo de comparar los tratamientos en cuanto a sus medias poblacionales. Este análisis de varianza (ANOVA) permite analizar a partir de una hipótesis, si el factor tiene o no impacto en la variable tiempo \cite{fallas}.

La estimación se realizó mediante el modelo siguiente.

\begin{equation}
 y{i}{j} = \mu_{1} + t_{i} + \varepsilon_{i,j}  
\end{equation}

en donde:
\begin{equation}
 \mu_{1}: \ \ \textup{representa la media de la población}
\end{equation}  
\begin{equation}
 t_{i}: \ \ \textup{representa el efecto del tratamiento i }
\end{equation}   
\begin{equation}
 \varepsilon_{i,j}: \ \ \textup{representa el error con una distribución normal}
\end{equation} 

Si el $p$ valor obtenido es menor que 0.05 se rechaza la hipótesis y esto quiere decir que hay diferencias entre las medianas, por lo que se realiza la prueba de \textit{Tukey} para mostrar las diferencias entre las medianas de los factores de \cite{tukey}.  

A continuación se muestran los cuadros que se obtuvieron con los resultados de la prueba estadística de ANOVA. 

\subsection{Influencia de la característica grado en el tiempo de ejecución} 

\begin{table}[H]
\caption{{\small Grado}}
%\resizebox{.5\textwidth}{!}{
\begin{center}		
	\centering
		\begin{tabular}{rrrrrrr}		
			\hline
			\textbf{Fuente} & \textbf{SS} & \textbf{DF} & \textbf{MS}& \textbf{F}& \textbf{$p$ valor} & \textbf{np2}\\
			
			\hline
			 Grado & 0.001 & 2 &0 & 22.123 & 0.003 & 0.023 \\
			 \textit{within}    & 0.032 & 1897 & 0 & ---- & ---- & ---- \\			 			 
			\hline
		\end{tabular}
		\label{cual}
	%\caption{Tabla de valores capturados}
	\label{tab:una-tablita}
\end{center}	
\end{table}

\begin{itemize}
  \item A partir del cuadro anterior se observa que existe diferencias entre el parámetro de las medianas de los grupos de factores ya que el $p$ valor es menor que 0.05, por lo que se rechaza la hipótesis de que el grado no influye en el tiempo de ejecución. Esto se puede observar en Figura 9 de la página 13. Debido a lo anterior se pasa a aplicar la prueba de \textit{Tukey} para mostrar las diferencias entre las medianas, lo que se evidencia en el cuadro 8 de la página 23.   
\end{itemize}


\subsection{Influencia de la característica coeficiente de agrupamiento en el tiempo de ejecución} 

\begin{table}[H]
\caption{{\small Coeficiente de agrupamiento}}
\begin{center}		
	\centering
		\begin{tabular}{rrrrrrr}		
			\hline
			\textbf{Fuente} & \textbf{SS} & \textbf{DF} & \textbf{MS}& \textbf{F}& \textbf{$p$ valor} & \textbf{np2}\\
			
			\hline
			 CoefAgrupamiento & 0.001 & 2 & 0.001 & 34.972 & 1.21e-106 & 0.036 \\
			 \textit{within}    & 0.032 & 1897 & 0  & --- & --- & --- \\			 			 
			\hline
		\end{tabular}
		\label{cual}
	%\caption{Tabla de valores capturados}
	\label{tab:una-tablita}
\end{center}	
\end{table}

\begin{itemize}
  \item Al observar el cuadro anterior se comprueba que existe diferencias entre el parámetro de las medianas de los grupos de factores ya que el $p$ valor es menor que 0.05, por lo que se rechaza la hipótesis de que el grado no influye en el tiempo de ejecución. Esto se puede observar en la Figura 10 de la página 14. A consecuencia de lo anterior se continua con la aplicación de la prueba de \textit{Tukey} para mostrar las diferencias entre las medianas, lo que se evidencia en el cuadro 9 de la página 24.   
\end{itemize}


\subsection{Influencia de la característica coeficiente de centralidad de cercanía en el tiempo de ejecución} 
\begin{table}[H]
\caption{{\small Coeficiente de centralidad de cercanía}}
\begin{center}		
	\centering
		\begin{tabular}{rrrrrrr}		
			\hline
			\textbf{Fuente} & \textbf{SS} & \textbf{DF} & \textbf{MS}& \textbf{F}& \textbf{$p$ valor} & \textbf{np2}\\
			
			\hline
			 Centralidad de cercania & 0 & 2 & 0 & 9.501 & 7.84e-05 & 0.01 \\
			 \textit{within}    & 0.032 & 1897 & 0  & --- & --- & --- \\			 			 
			\hline
		\end{tabular}
		\label{cual}
	%\caption{Tabla de valores capturados}
	\label{tab:una-tablita}
\end{center}	
\end{table}

\begin{itemize}
  \item Del cuadro anterior se puede observar que existe diferencias entre el parámetro de las medianas de los grupos de factores ya que el $p$ valor es menor que 0.05, por lo que se rechaza la hipótesis de que el grado no influye en el tiempo de ejecución. Esto es observable en la Figura 11 de la página 15. Como consecuencia de lo anterior, se pasa a aplicar la prueba de \textit{Tukey} para mostrar las diferencias entre las medianas, lo que se evidencia en el cuadro 10 de la página 24.   
\end{itemize}

\subsection{Influencia de la característica coeficiente de centralidad de carga en el tiempo de ejecución} 
\begin{table}[H]
\caption{{\small Coeficiente de centralidad de carga}}
\begin{center}		
	\centering
		\begin{tabular}{rrrrrrr}		
			\hline
			\textbf{Fuente} & \textbf{SS} & \textbf{DF} & \textbf{MS}& \textbf{F}& \textbf{$p$ valor} & \textbf{np2}\\
			
			\hline
			 centralidad de carga & 0.002 & 2 & 0.001 & 57.21 & 7.49e-25 & 0.057 \\
			 \textit{within}    & 0.031 & 1897 & 0  & --- & --- & --- \\			 			 
			\hline
		\end{tabular}
		\label{cual}
	%\caption{Tabla de valores capturados}
	\label{tab:una-tablita}
\end{center}	
\end{table}

\begin{itemize}
  \item En el cuadro anterior se muestra que no existe diferencias entre el parámetro de las medianas de los grupos de factores ya que el $p$ valor es menor que 0.05, por lo que se acepta la hipótesis de que el grado influye en el tiempo de ejecución. Esto se puede apreciar en la Figura 12 de la página 16.   
\end{itemize}

\subsection{Influencia de la característica coeficiente de excentricidad en el tiempo de ejecución} 

\begin{table}[H]
\caption{{\small Coeficiente de Excentricidad}}
\begin{center}		
	\centering
		\begin{tabular}{rrrrrrr}		
			\hline
			\textbf{Fuente} & \textbf{SS} & \textbf{DF} & \textbf{MS}& \textbf{F}& \textbf{$p$ valor} & \textbf{np2}\\
			
			\hline
			 Excentricidad & 0  & 2 & 0  & 10.451 & 3.06e-05 & 0.011 \\
			 \textit{within}    & 0.032 & 1897 & 0  & --- & --- & --- \\			 			 
			\hline
		\end{tabular}
		\label{cual}
	%\caption{Tabla de valores capturados}
	\label{tab:una-tablita}
\end{center}	
\end{table}

\begin{itemize}
  \item A partir del cuadro anterior se observa que existe diferencias entre el parámetro de las medianas de los grupos de factores ya que el $p$ valor es menor que 0.05, por lo que se rechaza la hipótesis de que el grado no influye en el tiempo de ejecución. Esto se puede corroborar al observar la Figura 13 de la página 17. Como consecuencia de lo anterior se continua con el siguiente paso,  aplicar la prueba de \textit{Tukey} para mostrar las diferencias entre las medianas, lo que se evidencia en el cuadro 12 de la página 24.   
\end{itemize}

\subsection{Influencia de la característica de pagerank en el tiempo de ejecución} 

\begin{table}[H]
\caption{{\small Coeficiente de PageRank}}
\begin{center}		
	\centering
		\begin{tabular}{rrrrrrr}		
			\hline
			\textbf{Fuente} & \textbf{SS} & \textbf{DF} & \textbf{MS}& \textbf{F}& \textbf{$p$ valor} & \textbf{np2}\\
			
			\hline
			 Excentricidad & 0.003  & 1 & 0.003  & 217.917 & 9.11e-47 & 0.103 \\
			 \textit{within}    & 0.029 & 1898 & 0  & --- & --- & --- \\			 			 
			\hline
		\end{tabular}
		\label{cual}
	%\caption{Tabla de valores capturados}
	\label{tab:una-tablita}
\end{center}	
\end{table}

\begin{itemize}
  \item Una vez observado el cuadro anterior se observa que no existe diferencias entre el parámetro de las medianas de los grupos de factores, ya que el $p$ valor es menor que 0.05, por lo que se acepta la hipótesis de que el grado influye en el tiempo de ejecución.    
\end{itemize}

\section{Prueba estadística}

En esta sección se presentan los resultados de la prueba estadística \textit{Tukey} o Método de \textit{Tukey} \cite{tukey}. Este método se utiliza luego de aplicar la prueba de ANOVA y está basada en una prueba de rango. Una prueba de ANOVA arroja si los resultados son significativos en general, pero no aclara dónde se encuentran esas diferencias. Luego se emplea el método conocido como Tukey para averiguar qué medios de grupos específicos son diferentes, o sea para crear intervalos de confianza para todas las diferencias en parejas entre las medias de los niveles de los factores y además controla la tasa de error por familia en un nivel especificado \cite{tukey}. %La figura de la página 6 \pageref{lastPage}} 
 

En esta prueba se construyen intervalos de confianza para todas las posibles comparaciones por parejas que sigue la forma:  

\begin{equation}
 \bar{(y_{1}} - \mu_{1}), \bar{(y_{2}} - \mu_{2}),... \bar{(y_{I}} - \mu_{I}) 
\end{equation}

Este método de\textit{Tukey} resuelve el contraste:
\begin{equation}
 H_{0}: \mu_{i}= \mu_{j} \ \ \textup{ vs }   H_{1}: \mu_{i} \neq \mu_{j} 
\end{equation}
\begin{equation}
 H_{0}: \ \ \textup{Hipótesis a demostrar}
\end{equation} 
\begin{equation}
 H_{1}: \ \ \textup{Hipótesis  alternativa}
\end{equation} 

Este contraste es la hipótesis que se demostrará si es verdadera o no.

A continuación se presentan los resultados del método de \textit{Tukey}.

Como se puede apreciar en el diagrama de caja existe diferencia entre las medianas y el Grado, el que más influye es el ...

\begin{figure}[H] 
    \centering
              \includegraphics[width=\textwidth]{images/boxplotGrado.eps}
\caption{Diagrama de caja del parámetro Grado}
\label{fig:seq1}
\end{figure}   

Como se puede apreciar en el diagrama de caja existe muy poca diferencia entre las medianas por lo que el tipo de .. en la variable dependiente.

\begin{figure}[H]
    \centering
        \includegraphics[width=\textwidth]{images/boxplotCoefAgrup.eps}        
\caption{Diagrama de caja de Coeficiente de agrupamiento y el tiempo}
\label{fig:seq1}
\end{figure}

Como se puede apreciar en el diagrama de caja existe diferencia entre las medianas, por lo que se puede concluir que la cantidad de nodos influye en la variable dependiente (el tiempo). 

\begin{figure}[H] 
    \centering
        \includegraphics[width=\textwidth]{images/boxplotCentralidadCe.eps}
\caption{Diagrama de caja de el coeficiente de centralidad de cercanía y el tiempo}
\label{fig:seq1}
\end{figure}

\begin{figure}[H] 
    \centering
        \includegraphics[width=\textwidth]{images/boxplotCentralidadCar.eps}
\caption{Diagrama de caja de el coeficiente de centralidad de carga y el tiempo}
\label{fig:seq1}
\end{figure}

\begin{figure}[H] 
    \centering
        \includegraphics[width=\textwidth]{images/boxplotExcentricidad.eps}
\caption{Diagrama de caja de el coeficiente de Excentricidad y el tiempo}
\label{fig:seq1}
\end{figure}

\begin{figure}[H] 
    \centering
        \includegraphics[width=\textwidth]{images/boxplotPageRank.eps}
\caption{Diagrama de caja de el coeficiente PageRank y el tiempo}
\label{fig:seq1}
\end{figure}


A continuación las gráficas de bigote del método \textit{Tukey}.

\begin{figure}[H] 
    \centering
        \includegraphics[width=\textwidth]{images/simultaneoustukeyGrado.eps}
\caption{Gráfica del coeficiente Grado y el tiempo}
\label{fig:seq1}
\end{figure} 

\begin{figure}[H] 
    \centering
        \includegraphics[width=\textwidth]{images/simultaneoustukeyCoefAgrup.eps}
\caption{Gráfica del coeficiente de agrupamiento y el tiempo}
\label{fig:seq1}
\end{figure} 

\begin{figure}[H] 
    \centering
        \includegraphics[width=\textwidth]{images/simultaneoustukeyCentralidadCe.eps}
\caption{Gráfica del coeficiente de centralidad de cercanía y el tiempo}
\label{fig:seq1}
\end{figure} 

\begin{figure}[H] 
    \centering
        \includegraphics[width=\textwidth]{images/simultaneoustukeyCentralidadCar.eps}
\caption{Gráfica del coeficiente de centralidad de carga y el tiempo}
\label{fig:seq1}
\end{figure} 

\begin{figure}[H] 
    \centering
        \includegraphics[width=\textwidth]{images/simultaneoustukeyExcentricidad.eps}
\caption{Gráfica del coeficiente de Excentricidad y el tiempo}
\label{fig:seq1}
\end{figure} 

A continuación le sigue el código para aplicar la prueba de ANOVA y luego la prueba estadística de \textit{Tukey}. La variable $tukey$ almacena los resultados de la prueba estadística. 

\subsection{Código}

\lstinputlisting[language=Python, firstline=8, lastline=90]{junto22ANOVA.py}

\section{Cuadros de la prueba de Tukey}

% Table generated by Excel2LaTeX from sheet 'TukeyGrado'
\begin{table}[htbp]
  \centering
  \caption{Ceoficiente de Grado}
    \begin{tabular}{llrrrl}
    group1 & group2 & \multicolumn{1}{l}{meandiff} & \multicolumn{1}{l}{lower} & \multicolumn{1}{l}{upper} & reject \\
          &       &       &       &       &  \\
    alta  & baja  & -0.0019 & -0.0026 & -0.0013 & True \\
          &       &       &       &       &  \\
    alta  & media & -0.0007 & -0.0013 & -0.0002 & True \\
          &       &       &       &       &  \\
    baja  & media & 0.0012 & 0.0006 & 0.0018 & True \\
    \end{tabular}%
  \label{tab:addlabel}%
\end{table}%

% Table generated by Excel2LaTeX from sheet 'TukeyCoefAgrup'
\begin{table}[htbp]
  \centering
  \caption{Coeficiente de Agrupamiento}
    \begin{tabular}{llrrrl}
    group1 & group2 & \multicolumn{1}{l}{meandiff} & \multicolumn{1}{l}{lower} & \multicolumn{1}{l}{upper} & reject \\
          &       &       &       &       &  \\
    alta  & baja  & 0.003 & 0.0007 & 0.0052 & True \\
          &       &       &       &       &  \\
    alta  & media & 0.0014 & -0.0008 & 0.0036 & False \\
          &       &       &       &       &  \\
    baja  & media & -0.0016 & -0.002 & -0.0011 & True \\
    \end{tabular}%
  \label{tab:addlabel}%
\end{table}%

% Table generated by Excel2LaTeX from sheet 'TukeyCentralidadCe'
\begin{table}[htbp]
  \centering
  \caption{Cuadro del coeficiente de Centralidad de cercanía}
    \begin{tabular}{llrrrl}
    group1 & group2 & \multicolumn{1}{l}{meandiff} & \multicolumn{1}{l}{lower} & \multicolumn{1}{l}{upper} & reject \\
          &       &       &       &       &  \\
    alta  & baja  & -0.001 & -0.0016 & -0.0005 & True \\
          &       &       &       &       &  \\
    alta  & media & -0.0004 & -0.001 & 0.0002 & False \\
          &       &       &       &       &  \\
    baja  & media & 0.0006 & 0.0001 & 0.0011 & True \\
    \end{tabular}%
  \label{tab:addlabel}%
\end{table}%



% Table generated by Excel2LaTeX from sheet 'TukeyCentralidadCar'
\begin{table}[htbp]
  \centering
  \caption{Cuadro del coeficiente de Centralidad de carga}
    \begin{tabular}{llrrrl}
    group1 & group2 & \multicolumn{1}{l}{meandiff} & \multicolumn{1}{l}{lower} & \multicolumn{1}{l}{upper} & reject \\
          &       &       &       &       &  \\
    alta  & baja  & -0.0016 & -0.0024 & -0.0008 & True \\
          &       &       &       &       &  \\
    alta  & media & 0.0006 & -0.0003 & 0.0015 & False \\
          &       &       &       &       &  \\
    baja  & media & 0.0022 & 0.0017 & 0.0027 & True \\
    \end{tabular}%
  \label{tab:addlabel}%
\end{table}%

% Table generated by Excel2LaTeX from sheet 'TukeyExcentricidad'
\begin{table}[htbp]
  \centering
  \caption{Coeficiente de Excentricidad}
    \begin{tabular}{llrrrl}
    group1 & group2 & \multicolumn{1}{l}{meandiff} & \multicolumn{1}{l}{lower} & \multicolumn{1}{l}{upper} & reject \\
          &       &       &       &       &  \\
    alta  & baja  & 0.0014 & 0.0006 & 0.0022 & True \\
          &       &       &       &       &  \\
    alta  & media & 0.0012 & 0.0005 & 0.0018 & True \\
          &       &       &       &       &  \\
    baja  & media & -0.0003 & -0.0008 & 0.0003 & False \\
    \end{tabular}%
  \label{tab:addlabel}%
\end{table}%


\section{Conclusión}

Se puede concluir luego del análisis del comportamiento de los datos reflejados en los cuadros y en las gráficas del método estadístico \textit{Tukey} que de los factores que son: el grado, el coeficiente de agrupamiento, el coeficiente de Centralidad de cercanía, el coeficiente de centralidad de carga, el coeficiente de excentricidad  y el coeficiente PageRank, los factores que más influyen en la variable dependiente (el tiempo de ejecución) son factor de el coeficiente de centralidad de carga y el coeficiente PageRank.

\bibliography{ref}
\bibliographystyle{plain}

\end{document}